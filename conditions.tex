\documentclass[a4paper]{article}

\usepackage{amsthm}

\usepackage{tikz}

\newtheorem{example}{Example}[section]

\title{Conditions}

\author{Johannes Marti and Leif Sabellek}

\begin{document}

\maketitle

\section{Simple path condition}

There is a subset $S$ of the pattern such that for each word $w$ there
is a node $s \in S$ which has a $w$-loop and such that there is a
$w$-path form $s$ to any other node $s' \in S$.

Example~2 shows that this condition is not sufficient.


\section{Automata condition}

For cofinitely many words $w$ there is a point $x$ such that every other
point can be reached from $x$ with a $w$-path.

Does not work because it does not force reflexive loops.


\section{More complex condition}

A point $x$ is a \emph{candidate} for $w$ if
\begin{itemize}
 \item $x$ sees every other point over $w$.
 \item if $w$ is a repetition of $v$ then $x$ has a $v$-loop.
\end{itemize}

The condition is that there are only finitely many words for which there
is no candidate.


\section{Easier version of the more complex condition}

For all words $w$ there is a point $x$ such that
\begin{itemize}
 \item $x$ sees every other point under $w^n$ for some $n$, and
 \item $x$ has a $w$ loop.
\end{itemize}

(I think that if the $0$ and $1$ relations in the pattern are right
total then this condition is equivalent to the previous condition. If
the relations are not we already know that there is no de Bruijn
homomorphism.)
This condition is co-c.e. The $n$ in the first bullet point can
be bound!

\end{document}
