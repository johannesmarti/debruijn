\documentclass[a4paper]{article}

\usepackage{a4wide}

\usepackage{amsmath}
\usepackage{amssymb}
\usepackage{amsthm}

\newcommand{\any}{{*}}
\newcommand{\size}[1]{{|{#1}|_s}}
\newcommand{\first}[1]{\mathsf{first}({#1})}
\newcommand{\last}[1]{\mathsf{last}({#1})}
\newcommand{\subpatterns}{\mathcal{S}}
\newcommand{\prearrow}[1]{{\rightarrow_{#1}^{-1}}}
\newcommand{\powerset}{\mathcal{P}}
\newcommand{\Q}{\mathcal{Q}}


%\usepackage{tikz}

\newtheorem{theorem}{Theorem}
\newtheorem{lemma}[theorem]{Lemma}
\newtheorem{proposition}[theorem]{Proposition}
\newtheorem{definition}[theorem]{Definition}

\newcommand{\case}[2]{\vspace{1ex}\noindent\textit{Case #1, #2:}}
%\newcommand{\case}[2]{\textbf{Case #1}, #2:}

\title{Construction Deterministic Patterns}

\author{Johannes Marti}

\begin{document}

\maketitle

\noindent The pattern $P$ is \emph{very strongly construction
deterministic} if for all $p,q \in P$ whenever $\prearrow{a}(p) \cap
\prearrow{a}(q) \neq \emptyset$ for all $a \in \Sigma$ then $p = q$.
Intuitively, this means that knowing both the 0- and 1-predecessor of
some node uniquely determines the node.

We weaken the notion of construction deterministic patterns to be
deterministic only up to a partition of the nodes in the pattern.

A \emph{partition} of a set $P$ is a surjective function $f : P \to P'$.
A partition $P$ is \emph{non-trivial} if $|P'| > 1$, that is, there is
more than one equivalence class.

A pattern $P$ is \emph{strongly construction deterministic} if there is
some non-trivial partition $f : P \to P'$ such that for all $p,q \in P$
and $a \in \Sigma$ if $f[\prearrow{a}(p)] \cap f[\prearrow{a}(q)] \neq
\emptyset$ then $f(p) = f(q)$.

One can weaken this notion even further: A pattern $P$ is
\emph{construction deterministic} if there is a family $\Q \subseteq
\powerset P$ with $P \subseteq \bigcup \Q$ and $P \notin \Q$ such that
for all $Q_0, Q_1 \in \Q$ there is a $Q \in \Q$ such that
\[
 [Q_0]{\rightarrow_0} \cap [Q_1]{\rightarrow_1} \subseteq Q.
\]
This notation means $[Q_a]{\rightarrow_a} = \{p \in P \mid q \rightarrow_a
p \mbox{ for some } q \in Q_a\}$ for both $a$.

It is a relatively easy observation that if $P$ is construction
deterministic as witnessed by some $\Q$ then this fact is also witnessed
by the downwards closure ${\Q\!\downarrow} = \{Q' \subseteq P \mid Q'
\subseteq Q \mbox{ for some } Q \in \Q\}$ of $\Q$ and by the anti-chain
of maximal elements of $\Q$ under the $\subseteq$ ordering.

It is also an easy observation that if a pattern is construction
deterministic then so is any sub pattern that has the same colors but
less edges.

\begin{lemma} \label{killer lemma}
 If $P$ is construction deterministic then there is no $k$ such
that there is a surjective homomorphism $T_k \to P$.
\end{lemma}
There seem to be two possible approaches to prove this Lemma. One is a
direct argument as Leif has suggested: We assume that there is a
surjective homomorphism and then show that the colors of all nodes need
to come from a class in $\Q$. The other approach is to show that in the
liftings we never construct lifted nodes whose underlying set is not
contained in some element of $\Q$.



\end{document}
