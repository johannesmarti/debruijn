\documentclass[a4paper]{article}

\usepackage{a4wide}

\usepackage{amsmath}
\usepackage{amssymb}
\usepackage{amsthm}

\newcommand{\powerset}{\mathcal{P}}


\newtheorem{theorem}{Theorem}
\newtheorem{lemma}[theorem]{Lemma}
\newtheorem{proposition}[theorem]{Proposition}
\newtheorem{definition}[theorem]{Definition}

\title{The Flying Pig}

\author{Johannes Marti}

\begin{document}

\maketitle

Let $\Sigma = \{a,b\}$. Order the set of all words from $\Sigma^*$ such
that $w \sqsubseteq v$ if either the length of $w$ is smaller than the
length of $v$ or if $w$ and $v$ are of the same length and $w$ comes
before $v$ in the obvious lexicographic order, assuming that a linear
order over $\Sigma$ is defined. Clearly, $\sqsubseteq$ is well-founded
linear order.

We define a set of words $U \subseteq \Sigma^*$ plus two trees $S$ and
$D$ over $U$. The root of both trees is the empty word $\epsilon$. We
write $S w v$ if $w$ is an ancestor of $v$ in the tree $S$. We write
$s(w)$ for the unique parent of $w \neq \epsilon$ in the tree $S$. We
use the analogous notations $D w v$ and $d(w)$ for the tree $D$.

The set $U$ and the trees $S$ and $D$ are defined in stages over $i \in
\omega$ such that at state $i$ we have already defined $(U_i,S_i,D_i)$
and are going to define $(U_{i + 1},S_{i + 1},D_{i + 1})$. such that
$U_i \subseteq U_{i + 1}$ and the trees $S_{i + 1}$ and $D_{i + 1}$ have
the same structure as $S_i$ and $D_i$ for all elements in $U_i$. The set
$U$ and the trees $S_i$ and $D_i$ are then the obvious limits of of
these objects.

In the inductive construction we maintain the following conditions as
invariants for $(U_i,S_i,D_i)$:
\begin{enumerate}
 \item A word $w \in U_i$ is a leaf of $S_i$ iff $w$ is a leaf of $D_i$.
\end{enumerate}

\end{document}
