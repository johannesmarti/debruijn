\documentclass[a4paper]{article}

\usepackage{a4wide}

\usepackage{amsmath}
\usepackage{amssymb}
\usepackage{amsthm}

\newcommand{\powerset}{\mathcal{P}}


\newtheorem{theorem}{Theorem}
\newtheorem{lemma}[theorem]{Lemma}
\newtheorem{proposition}[theorem]{Proposition}
\newtheorem{definition}[theorem]{Definition}

\title{The Flying Pig}

\author{Johannes Marti}

\begin{document}

\maketitle

Let $\Sigma = \{a,b\}$. Order the set of all words from $\Sigma^*$ such
that $w \sqsubseteq v$ if either the length of $w$ is smaller than the
length of $v$ or if $w$ and $v$ are of the same length and $w$ comes
before $v$ in the obvious lexicographic order, assuming that a linear
order over $\Sigma$ is defined. Clearly, $\sqsubseteq$ is well-founded
linear order.

We define a set of words $U \subseteq \Sigma^*$ plus two trees $S$ and
$D$ over $U$. The root of both trees is the empty word $\epsilon$. We
write $S w v$ if $w$ is an ancestor of $v$ in the tree $S$. We write
$s(w)$ for the unique parent of $w \neq \epsilon$ in the tree $S$. We
use the analogous notations $D w v$ and $d(w)$ for the tree $D$.

If $D u w$ then there is a path $v_0, \dots, v_n$ such that $v_0 = w$
and $v_n = u$ where $p(v_i) = v_{i + 1}$ holds for all $i < n$. In this
case we are interested in the word of first letters of the words along
this path. Thus, for all $u$ and $w$ with $D u w$ we define the word
$\pi(w,u) = a_0,\dots,a_{n - 1}$, where $a_i$ for $i < n$ is the first
letter of word $v_i$ on the path $v_0, \dots, v_n$ from $w$ to $u$ along
$p$. This is well-defined because $\epsilon$ is the root of $D$ and can
thus not be equal to any of the $v_i$ for $i < n$.

For every $w \in U \setminus\{\epsilon\}$ we consider the set $L(w)$ of
ancestors of $w$ in both the $S$ and $D$ relation. Thus we define
\[
 L(w) = \{ u \in U \mid S u w \mbox{ and } D u w\} .
\]
For all $w \in U \setminus \{\epsilon\}$ the set $C(w)$ is non-empty
because it always contains $\epsilon$, which is the root in both $S$ and
$D$. Because $D$ is a tree and $L(w)$ contains only $D$-ancestors of $w$
there is an element $l(w)$ in $L(w)$ that is closest in $D$ to $w$. More
formally, for all $w \in U \setminus \{\epsilon\}$ we let $l(w)$ be the
unique $u$ such that $u \in L(w)$ and $D u' u$ holds for all $u' \in
L(w)$ with $u' \neq u$. For each $w \in U \setminus \{\epsilon\}$ define
the word $\lambda(w) = \pi(w,l(w))$. Our aim will be to define $U$, $S$
and $D$ such that $\lambda(w) = w$ holds for all $w \in U \setminus
\{\epsilon\}$.

The set $U$ and the trees $S$ and $D$ are defined in stages over $i \in
\omega$ such that at state $i$ we have already defined $(U_i,S_i,D_i)$
and are going to define $(U_{i + 1},S_{i + 1},D_{i + 1})$. At every
stage we consider a leaf $w$ of $D_i$ and add either one or two children
of $w$ to $D_i$. The set $U$ and the trees $S$ and $D$ are then the
obvious limits of this construction.

\newcommand{\pred}{\mathsf{pred}}

In the inductive construction we maintain the following conditions as
invariants for $(U_i,S_i,D_i)$:
\begin{enumerate}
 \item Every word $w \in U_i$ occurs in both $S_i$ and $D_i$, meaning
that both $s_i(w)$ and $d_i(w)$ are defined if $w \neq \epsilon$.
% Maybe this condition is obvious and should not be mentioned
% explicitely.
 \item For every word $w \in U_i \setminus \{\epsilon\}$ we have that
$|s(w)| < |w|$.
 \item The first letters of $s(w)$ and $w$ are the same for all $w \in
U_i$ such that $s(w) \neq \epsilon$.
 \item For all $w \in U_i \setminus \{\epsilon\}$ we have that
$\lambda(w) = w$, where we take $\lambda$ to be defined in terms of
$S_i$ and $D_i$.
 \item For every $w \in U_i$ that is not a leaf in $D_i$ and letter $a
\in \Sigma$ there is a node $\pred_a(w) \in U_i$ that starts
with $a$ and is somehow nicely related to $w$.
\begin{enumerate}
 \item The first letter of $\pred_a(w)$ starts is $a$.
 \item If $\pred_a(s(w)) = \mathsf{turn}(w)$ then $\pred_a(w) =
\mathsf{turn}(w)$.
 \item If $\pred_a(s(w)) \neq \mathsf{turn}(w)$ then $d(\pred_a(w)) = w$.
\end{enumerate}
\end{enumerate}

In the base case of the inductive construction we define $U_i =
\{\epsilon, a, b\}$ and let both trees $S_0$ and $D_0$ be defined such
that $\epsilon$ is the root with children $a$ and $b$. It is easy to
check that this satisfies the conditions of the invariant.



\end{document}
