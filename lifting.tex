\documentclass[a4paper]{article}

\usepackage{a4wide}

\usepackage{amsmath}
\usepackage{amssymb}
\usepackage{amsthm}

\newcommand{\first}[1]{\mathsf{first}({#1})}
\newcommand{\last}[1]{\mathsf{last}({#1})}
\newcommand{\prearrow}[1]{{\rightarrow_{#1}^{-1}}}

\renewcommand{\iff}{\quad \mbox{iff} \quad}


\newtheorem{theorem}{Theorem}
\newtheorem{lemma}[theorem]{Lemma}
\newtheorem{proposition}[theorem]{Proposition}
\newtheorem{definition}[theorem]{Definition}

\title{The lifting of patterns}

\author{Johannes Marti}

\begin{document}

%\maketitle


\section{The definition}

Let $P$ be a pattern. The lifting $L(P)$ is a pattern that has as nodes
all subsets $V \subseteq P$ with $V \neq \emptyset$ and $|V| \leq 2$.
The relations are defined such that for both $a \in \{0,1\}$ we have
\[
 V \rightarrow_a W \iff \mbox{there is some } v \in V \mbox{ such that }
v \rightarrow_a w \mbox{ for all } w \in W.
\]
One can show that if $h : P \to Q$ is a homomorphism then $L(h) : L(P)
\to L(Q)$ with $L(h)(\{u,v\}) = \{h(u),h(v)\}$ is a homomorphism. Thus,
the lifting is a monotone operation on the homomorphism lattice. One can
also show that there is a homomorphism $e : P \to L(P)$ for all patterns
$P$. The homomorphism $e$ maps any $v \in P$ to $e(v) = \{v\}$.

The motivation for this notion is that one can show the following: For
all patterns $P$ and numbers $n$:
\begin{equation} \label{eq:lifting property}
 T_{n + 1} \to P \iff T_n \to L(P).
\end{equation}
Thus, it follows that $T_n \to P$ iff $T_0 \to L^n(P)$, where $L^n$
denotes the $n$-fold application of $L$. Note that $T_0 \to L^n(P)$ just
means that $L^n(P)$ has a point with reflexive $0$ and $1$-loops.


\section{The proof}

We are now going to prove \eqref{eq:lifting property}.
\begin{theorem}
 For all patterns $P$ and numbers $n \in \omega$  
\begin{equation*}
 T_{n + 1} \to P \iff T_n \to L(P).
\end{equation*}
\end{theorem}
\begin{proof}
 For the left-to-right direction assume that we have a homomorphism $h :
T_{n + 1} \to P$. We define that homomorphism $g : T_n \to L(P)$ by
setting $g(v) = \{h(v0),h(v1)\}$ for all $v \in \Sigma^n$. To see that
this defines a homomorphism assume that we have $v \rightarrow_b w$ in
$T_n$ for $v,w \in \Sigma^n$. We need to show that $\{h(v0),h(v1)\}
\rightarrow_b \{h(w0),h(w1)\}$ holds in $L(P)$. Consider $bw$ in $T_{n +
1}$. By definition of the de Bruijn graph $T_{n + 1}$ we have that $bw
\rightarrow_b w0$ and $bw \rightarrow_b w1$. It thus suffices to show
that there is some $c \in \Sigma$ for which $bw = vc$. If $n = 0$ then
we can simply set $c = b$. If $n > 0$ then it follows from $v
\rightarrow_b w$ that $v = b v'$ and $w = v' d$ for some $v' \in
\Sigma^{n - 1}$ and $d \in \Sigma$. We then have that $b v' d
\rightarrow_b w 0$ holds in $T_{n + 1}$. Because the $b$-predecessor of
$w 0$ is unique it follows that $b w = b v' d = v d$. So we can set $c =
d$.

 For the right-to-left direction assume that there is a homomorphism $g :
T_n \to L(P)$. We define a homomorphism $h : T_{n + 1} \to P$. To this
aim consider any $w c$ in $T_{n + 1}$ where $w \in \Sigma^n$ and $c \in
\Sigma$.

\end{proof}


\section{The plan}

The idea is to show that iterated applications of $L$ reaches a fixpoint
after finitely many steps. That is, we want to show that for every
pattern $P$ there is some computable $n$ such that $L^{n + 1}(P) \to
L^n(P)$. Because $P \to L(P)$ and $L$ is monotone we have that $L^{n +
1}(P) \leftarrow L^n(P)$, which entails means that $L^{n + 1} (P)$ and
$L^n(P)$ are homomorphically equivalent. Then, all larger $L^k(P)$ with
$k \geq n$ are also homomorphically equivalent to $L^n(P)$. We can then
check whether $T_0 \to L^k(P)$ for any $k$, by simply checking whether
$T_0 \to L^n(P)$.



\end{document}
