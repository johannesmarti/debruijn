\documentclass[a4paper]{article}

\usepackage{amsmath}
\usepackage{amssymb}
\usepackage{amsthm}

\newcommand{\any}{{*}}
\newcommand{\size}[1]{{|{#1}|_s}}
\newcommand{\first}[1]{\mathsf{first}({#1})}
\newcommand{\last}[1]{\mathsf{last}({#1})}
\newcommand{\subpatterns}{\mathcal{S}}

%\usepackage{tikz}

\newtheorem{theorem}{Theorem}[section]
\newtheorem{lemma}[theorem]{Lemma}
\newtheorem{proposition}[theorem]{Proposition}
\newtheorem{definition}[theorem]{Definition}

\newcommand{\case}[2]{\vspace{1ex}\noindent\textit{Case #1, #2:}}
%\newcommand{\case}[2]{\textbf{Case #1}, #2:}

\title{Three colors}

\author{Johannes Marti}

\begin{document}

\maketitle

\section{Notation}

Let $\Sigma = \{0,1\}$.

%Given a non-empty world $w = a u b$ for $a,b \in \Sigma$ and $u \in
%\Sigma^*$ we write $\first{w} = a$ and $\last{w} = b$ for its first and
%last letters.
%
%We write $s \rightarrow_i t$ for $i \in \{0,1\}$ if in the pattern there
%is an $i$-edge from $s$ to $t$.

\section{Case distinctions}

Let $P$ some pattern with three elements. Assume that there is a
homomorphism $h : T_k \to P$.

Let $a = h(0^\omega)$ and $b = h(1^\omega)$. Then $a \rightarrow_0 a$
and $b \rightarrow_1 b$. Note that $a \neq b$ because otherwise we have
a homomorphism $g : T_0 \to P, \star \mapsto a=b$. Let $c$ be the third
element of $P$ that is distinct from both $a$ and $b$. For a similar
reason we have that $a \not \rightarrow_1 a$ and $b \not \rightarrow_0
b$.

Note then that there is a $0$-path from $a$ to any other node in $P$ and
there is a $1$-path from $b$ to any other node in $P$. This means that
either $a \rightarrow_0 b,c$, $a \rightarrow_0 c \rightarrow_0 b$ or $a
\rightarrow_0 b \rightarrow_0 c$. Similarly, we have that either $b
\rightarrow_1 a,c$, $b \rightarrow_1 c \rightarrow_1 a$ or $b
\rightarrow_1 a \rightarrow_1 c$. Distinguish all possible combinations
of these cases:

\case{1}{$a \rightarrow_0 b,c$ and $b \rightarrow_1 a,c$} This case
closes because we get a homomorphism $g : T_1 \to P, 0 \mapsto a, 1
\mapsto b$.

\case{2}{$a \rightarrow_0 b,c$ and $b \rightarrow_1 c \rightarrow_1 a$}
In this case we have a homomorphism $g : T_2 \to P, 00,01 \mapsto a, 10
\mapsto c, 11 \mapsto b$.

\case{3}{$a \rightarrow_0 b,c$ and $b \rightarrow_1 a \rightarrow_1 c$}
This case closes because we get a homomorphism $g : T_1 \to P, 0 \mapsto
a, 1 \mapsto b$.


\case{4}{$a \rightarrow_0 c \rightarrow_0 b$ and $b \rightarrow_1 a,c$}
This case is symmetric to Case~2.

\case{5}{$a \rightarrow_0 c \rightarrow_0 b$ and $b \rightarrow_1 c
\rightarrow_1 a$}
We argue that in this case all the assumption of Lemma~\ref{first killer
lemma} are satisfied. We already know that $a \not \rightarrow_1 a$ and
$b \not \rightarrow_0 b$. Moreover, $c \not \rightarrow_0 c$ holds
because otherwise we have the homomorphism $g : T_1 \to P, 0 \mapsto c,
1 \mapsto b$. A symmetric argument shows that $c \not \rightarrow_1 c$.
To establish $a \not \rightarrow_0 b$ consider what happens if $a
\rightarrow_0 b$. Then we can construct the homomorphism $g : T_2 \to P,
00, 01 \mapsto a, 10 \mapsto c, 11 \mapsto b$. Symmetric reasoning shows
that $b \not \rightarrow_1 a$.

\case{6}{$a \rightarrow_0 c \rightarrow_0 b$ and $b \rightarrow_1 a
\rightarrow_1 c$}
We can establish all assumptions of Lemma~\ref{second killer lemma}. We
already know that $a \not \rightarrow_1 a$ and $b \not \rightarrow_0 b$.
If $a \rightarrow_0 b$ was true then we would have the homomorphism $T_1
\to P; 0 \mapsto a, 1 \mapsto b$. If $c \rightarrow_0 a$ was true then
we would have the homomorphism $T_3 \to P; 000,001,101 \mapsto a;
010,011 \mapsto c; 100,110,111 \mapsto b$. If $b \rightarrow_1 c$ was
the case then we would get the homomorphism $T_2 \to P; 00 \mapsto a; 01
\mapsto c; 10,11 \mapsto b$.


\case{7}{$a \rightarrow_0 b \rightarrow_0 c$ and $b \rightarrow_1 a,c$}
This case is symmetric to Case~3.

\case{8}{$a \rightarrow_0 b \rightarrow_0 c$ and $b \rightarrow_1 c
\rightarrow_1 a$}
This case is symmetric to Case~6.

\case{9}{$a \rightarrow_0 b \rightarrow_0 c$ and $b \rightarrow_1 a
\rightarrow_1 c$} This case closes because we get a homomorphism $g :
T_1 \to P, 0 \mapsto a, 1 \mapsto b$.

\section{Killer induction}

\begin{lemma} \label{first killer lemma}
 Assume $P$ has three elements $a$, $b$ and $c$ such that $a \not
\rightarrow_1 a$, $b \not \rightarrow_0 b$, $c \not \rightarrow_0 c$,
$c \not \rightarrow_1 c$, $a \not \rightarrow_0 b$ and $b \not
\rightarrow_1 c$. Then there is no $k$ such that $T_k \to P$.
\end{lemma}
\begin{proof}
 Assume for a contradiction that $h : T_k \to P$.

Clearly, $h(0^\omega) = a$.

Note first that the only $1$-predecessor of $a$ is $c$, the only
$0$-predecessor of $b$ is $c$ and \dots

We prove with an induction over $m$ that
\begin{enumerate}
 \item \label{even item} $\forall a_1 \exists x_1 \forall a_2 \dots
\forall a_m \exists x_m \  h(x_m a_m \dots a_2 x_1 a_1 1 0^\omega) = c$,
and
 \item \label{odd item} $\forall a_1 \exists x_1 \forall a_2 \dots
\forall a_m \exists x_m \forall a_{m + 1} \  h(a_{m + 1} x_m a_m \dots
a_2 x_1 a_1 1 0^\omega) \in \{a, b\}$.
\end{enumerate}
For the first case of the base case, where $m = 0$ observe that that
$h(10^\omega) = c$ because $h(10^\omega) \rightarrow_1 h(0^\omega)$. The
second part of the base case then follows because $c$ is neither a $0$
nor a $1$ predecessor of $c$. Thus, $h(a_1 1 0^\omega) \in \{a, b\}$ for
both $a_1 \in \{0,1\}$.

For the first part of the inductive step, where we prove the claims for
$m + 1$, observe first that by the inductive hypothesis
\[
 \forall a_1 \exists x_1 \forall a_2 \dots \forall a_m \exists x_m
\forall a_{m + 1} \  h(a_{m + 1} x_m a_m \dots a_2 x_1 a_1 1 0^\omega)
\in \{a, b\}.
\]
It follows that
\[
 \forall a_1 \exists x_1 \forall a_2 \dots \forall a_m \exists x_m
\forall a_{m + 1} \exists x_{m + 1} \  h(x_{m + 1} a_{m + 1} x_m a_m
\dots a_2 x_1 a_1 1 0^\omega) = c,
\]
because if $h(a_{m + 1} x_m a_m \dots a_2 x_1 a_1 1 0^\omega) = a$ then
its $1$-predecessor needs to be $c$ and if on the other hand $h(a_{m +
1} x_m a_m \dots a_2 x_1 a_1 1 0^\omega) = b$ then its $0$-predecessor
needs to be $c$. The second part of the inductive then follows because
neither $a$ nor $b$ is a $a_{m + 2}$-predecessor of $c$ for both $a_{m +
2} \in \{0,1\}$.

We then show that for all $m$ and $i \leq m$ and all $b_1,\dots,b_{2i}$
\begin{enumerate}
 \item \label{second power item} $\forall a_1 \exists x_1 \forall a_2
\dots \exists x_{m - i - 1} \forall a_{m - i} \  h(a_{m - i} x_{m - i - 1} \dots
a_2 x_1 a_1 1 0^\omega 1) \in \{a,b\}$, and
 \item \label{first power item} $\forall a_1 \exists x_1 \forall a_2
\dots \exists x_{m - i - 1} \forall a_{m - i} \exists x_{m - i} \
h(x_{m - i} a_{m - i} x_{m - i - i} \dots
a_2 x_1 a_1 1 0^\omega 1 b_1 b_{2i - 1}) = c$.
\end{enumerate}
To prove this we fix an arbitrary $m$ then then do an induction over $i
\leq m$.


\end{proof}

\begin{lemma} \label{second killer lemma}
 Assume $P$ has three elements $a$, $b$ and $c$ such that $a \not
\rightarrow_1 a$, $b \not \rightarrow_0 b$, $a \not \rightarrow_0 b$, $c
\not \rightarrow_0 a$, and $b \not \rightarrow_1 c$. Then there is no
$k$ such that $T_k \to P$.
\end{lemma}



\end{document}
