\documentclass[a4paper]{article}

\usepackage{a4wide}

\usepackage{amsmath}
\usepackage{amssymb}
\usepackage{amsthm}

\newcommand{\any}{{*}}
\newcommand{\size}[1]{{|{#1}|_s}}
\newcommand{\first}[1]{\mathsf{first}({#1})}
\newcommand{\last}[1]{\mathsf{last}({#1})}
\newcommand{\subpatterns}{\mathcal{S}}
\newcommand{\prearrow}[1]{{\rightarrow_{#1}^{-1}}}


%\usepackage{tikz}

\newtheorem{theorem}{Theorem}[section]
\newtheorem{lemma}[theorem]{Lemma}
\newtheorem{proposition}[theorem]{Proposition}
\newtheorem{definition}[theorem]{Definition}

\newcommand{\case}[2]{\vspace{1ex}\noindent\textit{Case #1, #2:}}
%\newcommand{\case}[2]{\textbf{Case #1}, #2:}

\title{Three colors}

\author{Johannes Marti}

\begin{document}

\maketitle

\section{Notation}

Let $\Sigma = \{0,1\}$.

%Given a non-empty world $w = a u b$ for $a,b \in \Sigma$ and $u \in
%\Sigma^*$ we write $\first{w} = a$ and $\last{w} = b$ for its first and
%last letters.
%
%We write $s \rightarrow_i t$ for $i \in \{0,1\}$ if in the pattern there
%is an $i$-edge from $s$ to $t$.

\section{Case distinctions}

Let $P$ some pattern with three elements. Assume that there is a
homomorphism $h : T_k \to P$.

Let $a = h(0^\omega)$ and $b = h(1^\omega)$. Then $a \rightarrow_0 a$
and $b \rightarrow_1 b$. Note that $a \neq b$ because otherwise we have
a homomorphism $g : T_0 \to P, \star \mapsto a=b$. Let $c$ be the third
element of $P$ that is distinct from both $a$ and $b$. For a similar
reason we have that $a \not \rightarrow_1 a$ and $b \not \rightarrow_0
b$.

Note then that there is a $0$-path from $a$ to any other node in $P$ and
there is a $1$-path from $b$ to any other node in $P$. This means that
either $a \rightarrow_0 b,c$, $a \rightarrow_0 c \rightarrow_0 b$ or $a
\rightarrow_0 b \rightarrow_0 c$. Similarly, we have that either $b
\rightarrow_1 a,c$, $b \rightarrow_1 c \rightarrow_1 a$ or $b
\rightarrow_1 a \rightarrow_1 c$. Distinguish all possible combinations
of these cases:

\case{1}{$a \rightarrow_0 b,c$ and $b \rightarrow_1 a,c$} This case
closes because we get a homomorphism $g : T_1 \to P, 0 \mapsto a, 1
\mapsto b$.

\case{2}{$a \rightarrow_0 b,c$ and $b \rightarrow_1 c \rightarrow_1 a$}
In this case we have a homomorphism $g : T_2 \to P, 00,01 \mapsto a, 10
\mapsto c, 11 \mapsto b$.

\case{3}{$a \rightarrow_0 b,c$ and $b \rightarrow_1 a \rightarrow_1 c$}
This case closes because we get a homomorphism $g : T_1 \to P, 0 \mapsto
a, 1 \mapsto b$.


\case{4}{$a \rightarrow_0 c \rightarrow_0 b$ and $b \rightarrow_1 a,c$}
This case is symmetric to Case~2.

\case{5}{$a \rightarrow_0 c \rightarrow_0 b$ and $b \rightarrow_1 c
\rightarrow_1 a$}
We argue that in this case leads to a weakly construction deterministic
pattern. We already know that $a \not \rightarrow_1 a$ and $b \not
\rightarrow_0 b$. Moreover, $c \not \rightarrow_0 c$ holds because
otherwise we have the homomorphism $g : T_1 \to P, 0 \mapsto c, 1
\mapsto b$. A symmetric argument shows that $c \not \rightarrow_1 c$. To
establish $a \not \rightarrow_0 b$ consider what happens if $a
\rightarrow_0 b$. Then we can construct the homomorphism $g : T_2 \to P,
00, 01 \mapsto a, 10 \mapsto c, 11 \mapsto b$. Symmetric reasoning shows
that $b \not \rightarrow_1 a$. The maximal pattern that we might have
now is as follows
\begin{align*}
 a,b,c & \rightarrow_0 a & c \rightarrow_1 a \\
 a,b & \rightarrow_0 c & a,b \rightarrow_1 c \\
 c & \rightarrow_0 b & a,c,b \rightarrow_1 b
\end{align*}
This pattern is weakly construction deterministic for the partition that
identifies $a$ with $b$.

\case{6}{$a \rightarrow_0 c \rightarrow_0 b$ and $b \rightarrow_1 a
\rightarrow_1 c$}
We can establish that $a \not \rightarrow_1 a$, $b \not \rightarrow_0
b$, $a \not \rightarrow_0 b$, $c \not \rightarrow_0 a$ and $b \not
\rightarrow_1 c$. We already know that $a \not \rightarrow_1 a$ and $b
\not \rightarrow_0 b$. If $a \rightarrow_0 b$ was true then we would
have the homomorphism $T_1 \to P; 0 \mapsto a, 1 \mapsto b$. If $c
\rightarrow_0 a$ was true then we would have the homomorphism $T_3 \to
P; 000,001,101 \mapsto a; 010,011 \mapsto c; 100,110,111 \mapsto b$. If
$b \rightarrow_1 c$ was the case then we would get the homomorphism $T_2
\to P; 00 \mapsto a; 01 \mapsto c; 10,11 \mapsto b$.

% Observe that $h((01)^\omega) \rightarrow_0 h((10)^\omega)$ and
% $h((10)^\omega) \rightarrow_1 h((00)^\omega)$. Clearly $h((10)^\omega)
% \neq h((00)^\omega)$ because otherwise we would have a coloring of
% $T_0$. We distinguish cases how $h((10)^\omega)$ and $h((00)^\omega)$
% distribute over $a$, $b$ and $c$, excluding all options that are
% impossible because of the connections that we already know to not exist.
% 
% \case{6.1}{$h((01)^\omega) = a$ and $h((10)^\omega) = c$} Thus, we get
% that $c \rightarrow_1 a$. It follows that $c \not \rightarrow_1 c$
% because otherwise we get a homomorphism $T_1 \to P; 0 \mapsto a; 1
% \mapsto c$.
% 
% \case{6.2}{$h((01)^\omega) = b$ and $h((10)^\omega) = a$} Thus, we get
% $b \rightarrow_0 a$ and $a \rightarrow_1 b$.
% 
% \case{6.2}{$h((01)^\omega) = b$ and $h((10)^\omega) = c$} Thus, we get
% $b \rightarrow_0 c$.


[other approach]

We distinguish
further cases depending on whether $c$ has no reflexive arrows, or it
is $1$- or $0$-reflexive.

\case{6.1}{$c \not \rightarrow_0 c$ and $c \not \rightarrow_1 c$}
This case closes because we obtain the construction deterministic pattern
\begin{align*}
 a,b & \rightarrow_0 a & c,b \rightarrow_1 a \\
 a,b & \rightarrow_0 c & a \rightarrow_1 c \\
 c & \rightarrow_0 b & a,c,b \rightarrow_1 b
\end{align*}

\case{6.2}{$c \rightarrow_0 c$ and $c \not \rightarrow_1 c$}

[open]

\case{6.3}{$c \not \rightarrow_0 c$ and $c \rightarrow_1 c$}

[open]

[end other approach]

\case{7}{$a \rightarrow_0 b \rightarrow_0 c$ and $b \rightarrow_1 a,c$}
This case is symmetric to Case~3.

\case{8}{$a \rightarrow_0 b \rightarrow_0 c$ and $b \rightarrow_1 c
\rightarrow_1 a$}
This case is symmetric to Case~6.

\case{9}{$a \rightarrow_0 b \rightarrow_0 c$ and $b \rightarrow_1 a
\rightarrow_1 c$} This case closes because we get a homomorphism $g :
T_1 \to P, 0 \mapsto a, 1 \mapsto b$.

\end{document}
