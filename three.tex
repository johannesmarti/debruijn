\documentclass[a4paper]{article}

\usepackage{amsmath}
\usepackage{amssymb}
\usepackage{amsthm}

\newcommand{\any}{{*}}
\newcommand{\size}[1]{{|{#1}|_s}}
\newcommand{\first}[1]{\mathsf{first}({#1})}
\newcommand{\last}[1]{\mathsf{last}({#1})}
\newcommand{\subpatterns}{\mathcal{S}}

%\usepackage{tikz}

\newtheorem{theorem}{Theorem}[section]
\newtheorem{lemma}[theorem]{Lemma}
\newtheorem{proposition}[theorem]{Proposition}
\newtheorem{definition}[theorem]{Definition}

\newcommand{\case}[2]{\vspace{1ex}\noindent\textit{Case #1, #2:}}
%\newcommand{\case}[2]{\textbf{Case #1}, #2:}

\title{Three colors}

\author{Johannes Marti}

\begin{document}

\maketitle

\section{Notation}

Let $\Sigma = \{0,1\}$.

%Given a non-empty world $w = a u b$ for $a,b \in \Sigma$ and $u \in
%\Sigma^*$ we write $\first{w} = a$ and $\last{w} = b$ for its first and
%last letters.
%
%We write $s \rightarrow_i t$ for $i \in \{0,1\}$ if in the pattern there
%is an $i$-edge from $s$ to $t$.

\section{Case distinctions}

Let $P$ some pattern with three elements. Assume that there is a
homomorphism $h : T_k \to P$.

Let $a = h(0^\omega)$ and $b = h(1^\omega)$. Then $a \rightarrow_0 a$
and $b \rightarrow_1 b$. Note that $a \neq b$ because otherwise we have
a homomorphism $g : T_0 \to P, \star \mapsto a=b$. Let $c$ be the third
element of $P$ that is distinct from both $a$ and $b$. For a similar
reason we have that $a \not \rightarrow_1 a$ and $b \not \rightarrow_0
b$.

Note then that there is a $0$-path from $a$ to any other node in $P$ and
there is a $1$-path from $b$ to any other node in $P$. This means that
either $a \rightarrow_0 b,c$, $a \rightarrow_0 c \righarrow_0 b$ or $a
\righarrow_0 b \rightarrow_0 c$. Similarly, we have that either $b
\rightarrow_1 a,c$, $b \rightarrow_1 c \righarrow_1 a$ or $b
\righarrow_1 a \rightarrow_1 c$. Distinguish all possible combinations
of these cases:

\case{1}{$a \rightarrow_0 b,c$ and $b \rightarrow_1 a,c$} This case
closes because we get a homomorphism $g : T_1 \to P, 0 \mapsto a, 1
\mapsto b$.


\section{Case distinctions}

Let $P$ some pattern with three elements. Assume that there is a
homomorphism $h : T_k \to P$.

Let $a = h(0^\omega)$ and $b = h(1^\omega)$. Then $a \rightarrow_0 a$
and $b \rightarrow_1 b$. Note that $a \neq b$ because otherwise we have
a homomorphism $g : T_0 \to P, \star \mapsto a=b$. Let $c$ be the third
element of $P$ that is distinct from both $a$ and $b$. For a similar
reason we have that $a \not \rightarrow_1 a$ and $b \not \rightarrow_0
b$.

Consider then $h((01)^\omega)$ and $h((10)^\omega)$, with
$h((01)^\omega) \rightarrow_0 h((10)^\omega)$ and $h((10)^\omega)
\rightarrow_1 h((01)^\omega)$. We have that $h((01)^\omega) \neq
h((10)^\omega)$ because otherwise there would be a homomorphism from
$T_0$. We then consider all the cases how $h((01)^\omega)$ and
$h((10)^\omega)$ can distribute over $a$, $b$ and $c$.

\case{1}{$a = h((01)^\omega)$ and $b = h((10)^\omega)$} This case closes
because we get a homomorphism $g : T_1 \to P, 0 \mapsto a, 1 \mapsto b$.

\case{2}{$b = h((01)^\omega)$ and $a = h((10)^\omega)$}

\case{3}{$a = h((01)^\omega)$ and $c = h((10)^\omega)$}

\case{4}{$c = h((01)^\omega)$ and $a = h((10)^\omega)$}

\case{5}{$b = h((01)^\omega)$ and $c = h((10)^\omega)$} This case is
symmetric to Case~4.

\case{6}{$c = h((01)^\omega)$ and $b = h((10)^\omega)$} This case is
symmetric to Case~3.

\end{document}
