\documentclass[a4paper]{article}

\usepackage{a4wide}

\usepackage{amsmath}
\usepackage{amssymb}
\usepackage{amsthm}

\newcommand{\any}{{*}}
\newcommand{\size}[1]{{|{#1}|_s}}
\newcommand{\first}[1]{\mathsf{first}({#1})}
\newcommand{\last}[1]{\mathsf{last}({#1})}
\newcommand{\subpatterns}{\mathcal{S}}
\newcommand{\prearrow}[1]{{\rightarrow_{#1}^{-1}}}


%\usepackage{tikz}

\newtheorem{theorem}{Theorem}[section]
\newtheorem{lemma}[theorem]{Lemma}
\newtheorem{proposition}[theorem]{Proposition}
\newtheorem{definition}[theorem]{Definition}

\newcommand{\case}[2]{\vspace{1ex}\noindent\textit{Case #1, #2:}}
%\newcommand{\case}[2]{\textbf{Case #1}, #2:}

\title{Three colors}

\author{Johannes Marti}

\begin{document}

\maketitle

\section{Notation}

Let $\Sigma = \{0,1\}$.

%Given a non-empty world $w = a u b$ for $a,b \in \Sigma$ and $u \in
%\Sigma^*$ we write $\first{w} = a$ and $\last{w} = b$ for its first and
%last letters.
%
%We write $s \rightarrow_i t$ for $i \in \{0,1\}$ if in the pattern there
%is an $i$-edge from $s$ to $t$.

\section{Case distinctions}

Let $P$ some pattern with three elements. Assume that there is a
homomorphism $h : T_k \to P$.

Let $a = h(0^\omega)$ and $b = h(1^\omega)$. Then $a \rightarrow_0 a$
and $b \rightarrow_1 b$. Note that $a \neq b$ because otherwise we have
a homomorphism $g : T_0 \to P, \star \mapsto a=b$. Let $c$ be the third
element of $P$ that is distinct from both $a$ and $b$. For a similar
reason we have that $a \not \rightarrow_1 a$ and $b \not \rightarrow_0
b$.

Note then that there is a $0$-path from $a$ to any other node in $P$ and
there is a $1$-path from $b$ to any other node in $P$. This means that
either $a \rightarrow_0 b,c$, $a \rightarrow_0 c \rightarrow_0 b$ or $a
\rightarrow_0 b \rightarrow_0 c$. Similarly, we have that either $b
\rightarrow_1 a,c$, $b \rightarrow_1 c \rightarrow_1 a$ or $b
\rightarrow_1 a \rightarrow_1 c$. Distinguish all possible combinations
of these cases:

\case{1}{$a \rightarrow_0 b,c$ and $b \rightarrow_1 a,c$} This case
closes because we get a homomorphism $g : T_1 \to P, 0 \mapsto a, 1
\mapsto b$.

\case{2}{$a \rightarrow_0 b,c$ and $b \rightarrow_1 c \rightarrow_1 a$}
In this case we have a homomorphism $g : T_2 \to P, 00,01 \mapsto a, 10
\mapsto c, 11 \mapsto b$.

\case{3}{$a \rightarrow_0 b,c$ and $b \rightarrow_1 a \rightarrow_1 c$}
This case closes because we get a homomorphism $g : T_1 \to P, 0 \mapsto
a, 1 \mapsto b$.


\case{4}{$a \rightarrow_0 c \rightarrow_0 b$ and $b \rightarrow_1 a,c$}
This case is symmetric to Case~2.

\case{5}{$a \rightarrow_0 c \rightarrow_0 b$ and $b \rightarrow_1 c
\rightarrow_1 a$}
We argue that in this case all the assumption of Lemma~\ref{first killer
lemma} are satisfied. We already know that $a \not \rightarrow_1 a$ and
$b \not \rightarrow_0 b$. Moreover, $c \not \rightarrow_0 c$ holds
because otherwise we have the homomorphism $g : T_1 \to P, 0 \mapsto c,
1 \mapsto b$. A symmetric argument shows that $c \not \rightarrow_1 c$.
To establish $a \not \rightarrow_0 b$ consider what happens if $a
\rightarrow_0 b$. Then we can construct the homomorphism $g : T_2 \to P,
00, 01 \mapsto a, 10 \mapsto c, 11 \mapsto b$. Symmetric reasoning shows
that $b \not \rightarrow_1 a$.

\case{6}{$a \rightarrow_0 c \rightarrow_0 b$ and $b \rightarrow_1 a
\rightarrow_1 c$}
We can establish that $a \not \rightarrow_1 a$, $b \not \rightarrow_0
b$, $a \not \rightarrow_0 b$, $c \not \rightarrow_0 a$ and $b \not
\rightarrow_1 c$. We already know that $a \not \rightarrow_1 a$ and $b
\not \rightarrow_0 b$. If $a \rightarrow_0 b$ was true then we would
have the homomorphism $T_1 \to P; 0 \mapsto a, 1 \mapsto b$. If $c
\rightarrow_0 a$ was true then we would have the homomorphism $T_3 \to
P; 000,001,101 \mapsto a; 010,011 \mapsto c; 100,110,111 \mapsto b$. If
$b \rightarrow_1 c$ was the case then we would get the homomorphism $T_2
\to P; 00 \mapsto a; 01 \mapsto c; 10,11 \mapsto b$.

Observe that $h((01)^\omega) \rightarrow_0 h((10)^\omega)$ and 
$h((10)^\omega) \rightarrow_1 h((00)^\omega)$. Clearly
$h((10)^\omega) \neq h((00)^\omega)$ because otherwise we would have a
coloring of $T_0$. We distinguish cases how $h((10)^\omega)$ and
$h((00)^\omega)$ distribute over $a$, $b$ and $c$, excluding all options
that are impossible because of the connections that wealready know to
not exist.

\case{6.1}{$h((01)^\omega) = a$ and $h((10)^\omega) = c$} Thus, we get
that $c \rightarrow_1 a$. It follows that $c \not \rightarrow_1 c$ because
otherwise we get a homomorphism $T_1 \to P; 0 \mapsto a; 1 \mapsto c$.

\case{6.2}{$h((01)^\omega) = b$ and $h((10)^\omega) = a$} Thus, we get
$b \rightarrow_0 a$ and $a \rightarrow_1 b$.

\case{6.2}{$h((01)^\omega) = b$ and $h((10)^\omega) = c$} Thus, we get
$b \rightarrow_0 c$.


[other approach:]

We distinguish
further cases depending on whether $c$ has no reflexive arrows, or it
is $1$- or $0$-reflexive.

\case{6.1}{$c \not \rightarrow_0 c$ and $c \not \rightarrow_1 c$}
This case closes because we obtain the deterministic pattern
\begin{align*}
 a,b & \rightarrow_0 a & c,b \rightarrow_1 a \\
 a,b & \rightarrow_0 c & a \rightarrow_1 c \\
 c & \rightarrow_0 b & a,c,b \rightarrow_1 b
\end{align*}

\case{6.2}{$c \rightarrow_0 c$ and $c \not \rightarrow_1 c$}

[open]

\case{6.3}{$c \not \rightarrow_0 c$ and $c \rightarrow_1 c$}

[open]


\case{7}{$a \rightarrow_0 b \rightarrow_0 c$ and $b \rightarrow_1 a,c$}
This case is symmetric to Case~3.

\case{8}{$a \rightarrow_0 b \rightarrow_0 c$ and $b \rightarrow_1 c
\rightarrow_1 a$}
This case is symmetric to Case~6.

\case{9}{$a \rightarrow_0 b \rightarrow_0 c$ and $b \rightarrow_1 a
\rightarrow_1 c$} This case closes because we get a homomorphism $g :
T_1 \to P, 0 \mapsto a, 1 \mapsto b$.

\section{Killer induction}

\begin{lemma} \label{first killer lemma}
 Assume $P$ has three elements $a$, $b$ and $c$ such that $a \not
\rightarrow_1 a$, $b \not \rightarrow_0 b$, $c \not \rightarrow_0 c$,
$c \not \rightarrow_1 c$, $a \not \rightarrow_0 b$ and $b \not
\rightarrow_1 a$. Then there is no $k$ such that $T_k \to P$.
\end{lemma}
\begin{proof}
 Assume for a contradiction that $h : T_k \to P$.

Clearly, $h(0^\omega) = a$.

Note first that the only $1$-predecessor of $a$ is $c$, the only
$0$-predecessor of $b$ is $c$ and \dots

\medskip

We prove with an induction over $n$ that
\begin{itemize}
 \item if $n$ is even then $\forall a_1 \exists x_2 \dots
\forall a_{n - 1} \exists x_n \  h(x_n a_{n - 1} \dots x_2 a_1 1 0^\omega) = c$,
and
 \item if $n$ is odd then $\forall a_1 \exists x_2
\dots \exists x_{n - 1} \forall a_n \  h(a_n x_{n - 1}\dots
x_2 a_1 1 0^\omega) \in \{a, b\}$.
\end{itemize}
For the base case, where $n = 0$ observe that $h(10^\omega) = c$ because
$h(10^\omega) \rightarrow_1 h(0^\omega)$. In the inductive step we
assume that the claim holds for $n$ and distinguish cases depending on
whether $n$ is even or odd.

If $n$ is even then we have by inductive hypothesis that
\[
 \forall a_1 \exists x_2 \dots \forall a_{n - 1} \exists x_n \  h(x_n
a_{n - 1} \dots x_2 a_1 1 0^\omega) = c.
\]
It follows that
\[
 \forall a_1 \exists x_2 \dots \forall a_{n - 1} \exists x_n \forall
a_{n + 1} \ h(a_{n + 1} x_n
a_{n - 1} \dots x_2 a_1 1 0^\omega) \in \{a,b\},
\]
because $c$ is is not a $a_{n + 1}$-predecessor of $c$ for either of
$a_{n + 1} \in \{0,1\}$.

If $n$ is odd then we have from the inductive hypothesis that
\[
\forall a_1 \exists x_2
\dots \exists x_{n - 1} \forall a_n \  h(a_n x_{n - 1}\dots
x_2 a_1 1 0^\omega) \in \{a, b\}.
\]
It follows that
\[
 \forall a_1 \exists x_2 \dots \exists x_{n - 1} \forall x_n \exists
x_{n + 1} \ h(e_{n + 1} a_n
e_{n - 1} \dots x_2 a_1 1 0^\omega) = c,
\]
because $c$ is the only $1$-predecessor of $a$ and $c$ is the only
$0$-predecessor of $b$.

\medskip

We then show that for all $n$, $i \leq n$ and all $b_1,\dots,b_i$
\begin{itemize}
 \item if $n - i$ is even then $\forall a_1 \exists x_2 \dots \forall
a_{n - i - 1} \exists e_{n - i} \  h(e_{n - i} a_{n - i - 1} \dots x_2
a_1 1 0^\omega b_1 \dots b_i) = c$, and
 \item if $n - i$ is odd then $\forall a_1 \exists x_2
\dots \exists x_{n - i - 1} \forall a_{n - i} \
h(a_{n - i} x_{n - i - 1} \dots
x_2 a_1 1 0^\omega b_1 \dots b_i) \in \{a,b\}$.
\end{itemize}
To prove this, fix an arbitrary $n$ and then do an induction over $i
\leq n$. The base case where $i = 0$ is precisely the claim from the
previous induction. In the inductive step we assume the claim holds for
$i$ and distinguish cases depending on whether $n - i$ is even or odd.

If $n - i$ is even then we have as inductive assumption that for all
$b_1,\dots,b_i$
\[
 \forall a_1 \exists x_2 \dots \forall a_{n - i - 1} \exists e_{n - i} \
h(e_{n - i} a_{n - i - 1} \dots x_2 a_1 1 0^\omega b_1 \dots b_i) = c.
\]
We show that for all $b_1,\dots,b_i,b_{i + 1}$
\[
 \forall a_1 \exists x_2 \dots \exists x_{n - i - 2} \forall a_{n - i -
1} \ h(a_{n - i - 1} x_{n - i - 2} \dots x_2 a_1 1 0^\omega b_1 \dots
b_i b_{i + 1}) \in \{a,b\}.
\]
To this aim fix $b_1,\dots,b_i,b_{i + 1}$. By the inductive hypothesis
we have that for all $a_1$ there is a $x_2$ such that \dots there is
$x_{n - i - 2}$ such that for all $a_{n - i - 1}$ we have $\exists e_{n
- i} \ h(e_{n - i} a_{n - i - 1} \dots x_2 a_1 1 0^\omega b_1 \dots b_i)
  = c$. This means that the $e_{n - i}$-predecessor of $a_{n - i - 1}
x_{n - i - 2} \dots x_2 a_1 1 0^\omega b_1 \dots b_i b_{i + 1}$ maps to
$c$. Because $c \not \rightarrow_{e_{n - i}} c$ it follows that $h(a_{n
- i - 1} x_{n - i - 2} \dots x_2 a_1 1 0^\omega b_1 \dots b_i b_{i + 1})
\in \{a,b\}$.

If $n - i$ is odd then we have as inductive assumption that for all
$b_1,\dots,b_i$
\[
 \forall a_1 \exists x_2
\dots \exists x_{n - i - 1} \forall a_{n - i} \
h(a_{n - i} x_{n - i - 1} \dots
x_2 a_1 1 0^\omega b_1 \dots b_i) \in \{a,b\}.
\]
We show that for all $b_1,\dots,b_i,b_{i + 1}$
\[
 \forall a_1 \exists x_2 \dots \forall a_{n - i - 2} \exists x_{n - i -
1} \ h(e_{n - i - 1} a_{n - i - 2} \dots x_2 a_1 1 0^\omega b_1 \dots
b_i b_{i + 1}) = c.
\]
To this aim fix $b_1,\dots,b_i,b_{i + 1}$. By the inductive hypothesis
we have that for all $a_1$ there is a $x_2$ such that \dots there is
$x_{n - i - 2}$ such that for all $a_{n - i - 1}$ we have $\forall a_{n
- i} \ h(a_{n - i} e_{n - i - 1} \dots x_2 a_1 1 0^\omega b_1 \dots b_i)
  \in \{a,b\}$. This means that both predecessors of $x_{n - i - 1} a_{n
- i - 2} \dots x_2 a_1 1 0^\omega b_1 \dots b_i b_{i + 1}$ map to either
$a$ or $b$, but not to $c$. It follows that $h(a_{n - i - 1} x_{n - i -
2} \dots x_2 a_1 1 0^\omega b_1 \dots b_i b_{i + 1}) = c$, because $a$
has no $1$ predecessor among $\{a,b\}$ and $b$ has no $0$-predecessor
among $\{a, b\}$.

\medskip

We finally show that for all $n$ and $b_1,\dots,b_n$
\[
 h(0^\omega b_1,\dots b_n) = a.
\]
The base case where $n = 0$ holds because $h(0^\omega) = a$.
In the inductive step we assume that for all $b_1,\dots,b_n$
\[
 h(0^\omega b_1,\dots b_n) = a.
\]
We choose $b_1,\dots,b_n,b_{n + 1}$ and need to show that
\[
 h(0^\omega b_1,\dots b_n b_{n + 1}) = a.
\]
We can think of the inductive hypothesis as telling us that the
$0$-predecessor of $0^\omega b_1,\dots b_n b_{n + 1}$ has color $a$.
From the previous induction for the case where the $n$ from the
statement of the previous induction is the same as our current $n$ and
the $i$ from the statement of the previous induction is also $n$ we get
that
\[
 h(1 0^\omega b_1,\dots b_n) = c.
\]
This tells us that the $1$-predecessor of $0^\omega b_1,\dots b_n b_{n +
1}$ has color $c$. It follows that $0^\omega b_1,\dots b_n b_{n +
1}$ has color $a$ because $a \not \rightarrow_0 c$ and $c \not
\rightarrow_1 c$.

\medskip

We have shown that all nodes that are reachable over a $0$-path from
$0^\omega$ need to have color $a$. But since all nodes in $T_k$ are
reachable by such a path we have a contradiction because there is some
node in $T_k$, for instance $1^\omega$, that can not have color $a$.
\end{proof}


\section{Construction deterministic patterns}

The pattern $P$ is \emph{construction deterministic} if for all $p,q \in
P$ and $a \in \Sigma$ if $\prearrow{a}(p) \cap \prearrow{a}(q) \neq
\emptyset$ then $p = q$. Intuitively, this means that knowing both the
0- and 1-predecessor of some node uniquely determines the node.

We weaken the notion of construction deterministic patterns to be
deterministic only up to a partition of the nodes in the pattern.

A \emph{partition} of a set $P$ is a surjective function $f : P \to P'$.
A partition $P$ is \emph{non-trivial} if $|P'| > 1$, that is, there is
more than one equivalence class.

A pattern $P$ is \emph{weakly construction deterministic} if there is
some non-trivial partition $f : P \to P'$ such that for all $p,q \in P$
and $a \in \Sigma$ if $f[\prearrow{a}(p)] \cap f[\prearrow{a}(q)] \neq
\emptyset$ then $f(p) = f(q)$.

\begin{lemma} \label{killer lemma}
 If $P$ is weakly construction deterministic then there is no $k$ such
that there is a surjective homomorphism $T_k \to P$.
\end{lemma}



\end{document}
